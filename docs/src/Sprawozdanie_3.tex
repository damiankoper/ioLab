\documentclass[12pt]{article}
\usepackage{scrextend}
\usepackage[utf8]{inputenc}
\usepackage[polish]{babel}
\usepackage[T1]{fontenc}%polskie znaki
\usepackage[utf8]{inputenc}%polskie znaki
\usepackage{geometry}
\usepackage{float}
\usepackage{enumitem}
\usepackage{hyperref}
\usepackage{graphicx}
\usepackage{tabulary}
\usepackage{etoc}
\usepackage[normalem]{ulem} 
\renewcommand{\baselinestretch}{1.5}
\graphicspath{ {img/} }
\newgeometry{lmargin=2cm, rmargin=2cm, tmargin=2cm, bmargin=2cm}
\usepackage{tikz}
\usepackage[bf]{caption}
\usepackage{setspace}
\newcommand{\scenario}[5]{
  \subsection{#1}
  \begin{minipage}{\textwidth}
    \textbf{Cel:} #2 \\
    \textbf{WS:} #3 \\
    \textbf{WK:} #4 \\
    \textbf{Przebieg:}
    \begin{enumerate}
      \setlength\itemsep{0.0em}
      #5
    \end{enumerate}
  \end{minipage}
}
\hyphenation{include}
\hyphenation{extend}

\begin{document}

\begin{flushleft}
        Damian Koper, \textbf{241292} \\
        Łukasz Handschuh, \textbf{241402}
\end{flushleft}
\vspace{1cm}
{
    \centering
    {\Huge\scshape\bfseries Inżynieria oprogramowania - Etap 3 }\\
    \vspace{0.25cm}
    \Large\textbf{Dział ewidencji ludności} \\
    \vspace{0.25cm}
    \large Budowa diagramu czynności reprezentującego model
    biznesowy „świata rzeczywistego” na podstawie
    wykonanego opisu procesów biznesowych. Budowa
    diagramów czynności reprezentujących scenariusze
    wybranych przypadków użycia.\\
}

\section{PU Wyświetlanie wniosków}



\section{PU Edycja danych wniosku}



\end{document}