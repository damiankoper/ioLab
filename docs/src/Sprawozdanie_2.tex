\documentclass[12pt]{article}
\usepackage{scrextend}
\usepackage[utf8]{inputenc}
\usepackage[polish]{babel}
\usepackage[T1]{fontenc}%polskie znaki
\usepackage[utf8]{inputenc}%polskie znaki
\usepackage{geometry}
\usepackage{float}
\usepackage{enumitem}
\usepackage{hyperref}
\usepackage{graphicx}
\usepackage{tabulary}
\usepackage{etoc}
\usepackage[normalem]{ulem} 
\renewcommand{\baselinestretch}{1.5}
\graphicspath{ {img/} }
\newgeometry{lmargin=2cm, rmargin=2cm, tmargin=2cm, bmargin=2cm}
\usepackage{tikz}
\usepackage[bf]{caption}
\newcommand{\scenario}[5]{
  \subsection{#1}
  \begin{minipage}{\textwidth}
    \textbf{Cel:} #2 \\
    \textbf{WS:} #3 \\
    \textbf{WK:} #4 \\
    \textbf{Przebieg:}
    \begin{enumerate}
      \setlength\itemsep{0.0em}
      #5
    \end{enumerate}
  \end{minipage}
}

\begin{document}

\begin{flushleft}
        Damian Koper, \textbf{241292} \\
        Łukasz Handschuh, \textbf{241402}
\end{flushleft}
\vspace{1cm}
{
    \centering
    {\Huge\scshape\bfseries Inżynieria oprogramowania - Etap 2 }\\
    \vspace{0.25cm}
    \Large\textbf{Dział ewidencji ludności} \\
    \vspace{0.25cm}
    \large Specyfikacja wymagań funkcjonalnych za pomocą diagramu
    przypadków użycia.\\
}

\section{Wymagania funkcjonalne}

\subsection{Pracownicy}
\begin{enumerate}
    \item Pracownik może zalogować się na swoje konto i z niego się wylogować.
    \item Pracownik z rolą administratora może zarządzać kontami użytkowników.
    \begin{enumerate}
        \item Administrator może stworzyć konto użytkownika.
        \item Administrator może edytować dane i jego rolę użytkownika.
        \item Administrator może usunąć konto użytkownika.
    \end{enumerate}
\end{enumerate}
\subsection{Zarządzanie wnioskami}
\begin{enumerate}
    \item Pracownik może wyświetlić dane wniosków.
    \begin{enumerate}
        \item Pracownik może wyświetlić wnioski tylko danego typu.
        \item Pracownik może wyszukać wniosek podając nr. PESEL.
    \end{enumerate}
    \item Pracownik może wprowadzić wniosek meldunkowy.
    \begin{enumerate}
        \item Pracownik może wprowadzić wniosek o typie \textit{stały} lub \textit{czasowy}.
    \end{enumerate}
    \item Pracownik może edytować dane wniosku meldunkowego.
    \begin{enumerate}
        \item Pracownik może zmienić status wniosku - zaakceptować, lub odrzucić po podaniu powodu.
        \item Zaakceptowany wniosek zmienia się w rekord meldunku.
    \end{enumerate}
    \item Wniosek, by być możliwym do zaakceptowania, musi zawierać komplet danych, które zgadzają się z danymi z systemu PESEL.
\end{enumerate}
\subsection{Zarządzanie meldunkami}
\begin{enumerate}
    \item Pracownik może wyświetlić dane meldunków.
    \begin{enumerate}
        \item Pracownik może wyświetlić dane meldunków tylko danego typu.
        \item Pracownik może wyszukać dane meldunku podając nr. PESEL.
    \end{enumerate}
    \item Pracownik może edytować dane meldunków.
    \begin{enumerate}
        \item Pracownik może zmienić status meldunku na przeszły.
    \end{enumerate}
\end{enumerate}
\section{Wymagania niefunkcjonalne}
\begin{enumerate}
    \item Wszyscy użytkownicy, którzy mają dostęp do aplikacji (posiadają indywidualne konto) mają uprawnienia do edycji danych meldunkowych.
    \item Rola administratora pozwala na zarządzanie użytkownikami i ich uprawnieniami.
    \item System rejestruje historię wszystkich zmian danych meldunkowych.
    \item Dane przesyłane pomiędzy aplikacją, a systemem PESEL są szyfrowane.
    \item W przypadku braku łączności z systemem PESEL pracownik otrzymuje ostrzeżenie o tym fakcie.
    \item Podstawowym źródłem dostępu do danych jest aplikacja internetowa.
    \item Aplikacja umożliwia dostęp do danych przez podstawowy interface konsoli.
\end{enumerate}
\newpage
\section{Diagram przypadków użycia}

\section{Scenariusze przypadków użycia}

\scenario
    {xd}
    {xdd}
    {xddd}
    {xdddd}
    {
        \begin{enumerate}
            \setlength\itemsep{0.1em}
            \item jkjasadksad
            \item nknaasdnklasd
            \item asknksad
        \end{enumerate}
    }

\end{document}